\documentclass{article}

\usepackage{amsmath}                                  
\usepackage[margin=1in]{geometry}
\usepackage{graphicx}
\usepackage{bm}

\setlength\parindent{0pt}

% commands
\newcommand{\R}{\textbf{R}}
\newcommand{\eps}{\epsilon}
\newcommand{\norm}[1]{ \lVert #1 \rVert }



\begin{document}
\textbf{Some solutions to Kreyszig's
  \textit{Introductory Functional Analysis with Applications}, chapter 2
}

\

\textbf{2.1.6}

Show that in an $n$-dimensional vector space $X$,
the representation of any $x$ as a linear combination of given basis
vectors $e_1, \dots, e_n$ is unique.

\

Suppose we have two representations for $x$:
\begin{align*}
  & x = \alpha_1e_1 + \dots \alpha_n e_n \\
  \text{and } & x = \beta_1e_1 + \dots + \beta_n e_n \\
\iff & 0 = x-x = (\alpha_1-\beta_1)e_1 + \dots + (\alpha_n-\beta_n) e_n 
\end{align*}
Since the basis vectors must be linearly independent,
this means that each $(\alpha_i-\beta_i) = 0$,
i.e. $\alpha_i = \beta_i$.

\

\textbf{2.1.10}

If $Y$ and $Z$ are subspaces of a vector space $X$,
show that $Y \cap Z$ is a subspace of $X$,
but $Y \cup Z$ need not be one. 
Give examples. 

\

For $Y \cap Z$,
we show that this set is closed under scalar multiplication and addition. 
Given $x \in Y\cap Z$,
we know $\alpha x \in Y$ since $Y$ is a vector space,
and $\alpha x \in Z$ since $Z$ is a vector space,
hence $\alpha x \in Y \cap Z$. 
Similarly, 
given any $x$ and $y$ in $Y \cap Z$,
$x+y \in Y$ by definition of subspace for $Y$ and
$x+y \in Z$ by definition of subspace for $Z$,
hence $x+y \in Y \cap Z$. 

For $Y \cup Z$,
we can use $\R^2$ as a counterexample.
Let the ``$x$-axis'' or span\{(1,0)\} act as $Z$ and the
``$y$-axis'' or span\{(0,1)\} act as $Y$.
The union $Y \cup Z$ is not closed under vector addition
(note that $(0,1) \in Y$ and $(1,0) \in Z$): 
$$(1,0) + (0,1) = (1,1) \notin Y \cup Z$$

Clearly the intersection of these two sets, the zero vector, is a subspace. 

\

\textbf{2.2.11}

A subset $A$ of a vector space $X$ is said to be \textit{convex}
if $x,y \in A$ implies
$$ M = \{z \in X : z = \alpha x + (1-\alpha) y, 0 \leq \alpha \leq 1 \} \subset A$$
$M$ is called a closed segment with boundary points $x$ and $y$;
any other $z\in M$ is called an interior point of $M$.
Show that the closed unit ball
$$ \tilde{B}(0;1) = \{x\in X : \lVert x \rVert \leq 1 \}$$
in a normed space $X$ is convex. 

\

We take any two points $x$ and $y$ in the unit ball $\tilde{B}(0;1)$ 
and show that any point in the segment joining $x$ and $y$ is also
in $\tilde{B}(0;1)$.
Let $z \in M$, the line segment.
Then there exists $\alpha \in [0,1]$ such that
\begin{align*}
  z &= \alpha x + (1-\alpha) y \\
  \text{so \quad} \norm{z} &= \norm{ \alpha x + (1-\alpha) y } \\
  \text{triangle inequality: \quad} & \leq \norm{ \alpha x } + \norm{ (1-\alpha) y } \\
  \text{scalar invariance of norm: \quad} &= |\alpha| \norm{ x } + |(1-\alpha)|  \norm{ y } \\
  \text{$\norm{x}, \norm{y} \leq 1$: \quad} &\leq |\alpha| + |1-\alpha| \\
  \text{$0 \leq \alpha \leq 1$: \quad} & = \alpha + 1 - \alpha = 1
\end{align*}
So $\norm{z} \leq 1$ so it must be in the unit ball $\tilde{B}(0;1)$.

\

This is easy enough to see for $\R^2$ but what about for more exotic
spaces like C[0,1]?
This means that if we combine any two continuous functions with a common bound 
in the manner above for $z$,
the result will have the same bound.
See jupyter notebook \verb+lineoffunctions.ipynb+
for an illustrating animation. 

\

\textbf{2.3.2} Show that $c_0$, the space of all sequences of scalars
converging to zero, is a closed subspace of $l^\infty$. 

\

We want to show that the complement of $c_0$ is open.
That is,
every element in the set of all non-convergent sequences 
has a neighborhood of only non-convergent sequences.
First write what it means for an $x = \{x_1, x_2, \dots\} \in c_0^C$
not to converge to zero: 
there is an $\epsilon > 0$ such that for all $N$,
there is an $n > N$ such that $|x_n-0| \geq \epsilon$.
Now let an $x' = \{x_1', x_2', ...\} \in B(x,\epsilon/2)$
for the same $\epsilon$,
so we have from the metric on $l^\infty$ that 
$\sup |x_i - x_i'| < \epsilon/2 $.
Now we want to show that $x'$ does not converge.
From non-convergence of $x$, 
\begin{align*}
  \epsilon &\leq |x_n| = |x_n - x_n' + x_n'| \\
  \text{ triangle inequality: \quad} &\leq |x_n - x_n'| + |x_n'| \\
  \text{ by construction of $x'$: \quad} &< \eps/2 + |x_n'| \\
  \text{ subtract $\epsilon$/2 from first and last expression: \quad}
  \epsilon/2 &< |x_n'|
\end{align*}
hence there is an $\eps' = \eps/2$ so that $\forall N$ there is an $n>N$
so that $|x_n' - 0| \geq \eps'$, 
which means that $x'$ does not converge to zero. 
Since $x'$ was an arbitrary element of an open ball of $x$,
and $x\in c_0^C$ is arbitrary,
$c_0^C$ is open, 
which means $c_0$ is closed.

\

I had a crisis of faith.
For some reason I thought that the monomials formed a Cauchy sequence. 
But their limit is discontinuous and not a polynomial. 
The limiting function equals zero on $[0,1)$ and 1 at $t=1$. 
Since we showed that $C[a,b]$ is complete, 
this is a contradiction,
so it must not be Cauchy. 
Let's prove this directly just for fun.

\

Let $\epsilon = 1/2$ and take any positive integer $N$.
We want to show that there exist $n,m > N$ with
$\sup |t^n - t^m| \geq \epsilon$.
It's sufficient to show that there exists $t \in [0,1]$
such that $|t^n - t^m| \geq \epsilon$. 
For all $n>1$, $t^n$ takes all values in [0,1) for $t \in [0,1)$. 
Let $n=N+1$ and $t_0$ denote the input such that $t_0^n = 3/4$. 
Now we know that the monomials converge to zero pointwise on $[0,1)$
so 
for $\epsilon' = 1/4$ there exists $N'$ such that $m'>N'$ implies 
$|t_0^{m'}| < \epsilon'$. Let $m=\max\{N,N'\}+1$ so we still have $|t_0^{m}| < \epsilon'$.
Hence 
\begin{align*}
  |t_0^m - t_0^n| = t_0^n - t_0^m = 3/4 - t_0^m > 3/4 - 1/4 = 1/2
\end{align*}
So for some positive $\epsilon$,
for arbitrary $N$ there exists $n,m>N$ with
$\sup |t^m - t^n| \geq \epsilon$,
which means $\{t^n\}$ is not Cauchy under $\sup|t^n - t^m|$. 

\

\textbf{Alternative proof of Corollary 2.7-10 (b)}

The text shows that the null space of a bounded linear operator
is closed by showing that the closure of the null space is the same set. 
I would like a proof using the definition of a closed set,
i.e. that the complement is open.

\

Let $x \in N(T)^C$, the complement of the null space.
We want a neighborhood of $x$ fully contained in $N(T)^C$. 
First, since $T$ is bounded, it is also continuous. 
Hence for $\epsilon = \norm{Tx}/2$, 
there exists a $\delta$ such that any $x'$ 
satisfying ${\norm{x - x'}<\delta}$ (i.e. ${x' \in B(x,\delta)}$)
also has ${\norm{Tx - Tx'} < \epsilon}$.
Now take an arbitrary such $x'$
and suppose it is in the null space
(in order to show a contradiction),
i.e. $Tx = 0$. 
We have
\begin{align*}
  \frac{\norm{Tx}}{2} &< \norm{Tx} \\
  &= \norm{Tx - Tx'} \\
  \text{and by continuity of $T$:\quad} &< \epsilon = \frac{\norm{Tx}}{2}
\end{align*}
a contradiction. 

\end{document}
