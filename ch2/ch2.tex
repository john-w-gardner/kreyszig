\documentclass{article}

\usepackage{amsmath}                                  
\usepackage[margin=1in]{geometry}
\usepackage{graphicx}
\usepackage{bm}

\setlength\parindent{0pt}

% commands
\newcommand{\R}{\textbf{R}}
\newcommand{\norm}[1]{ \lVert #1 \rVert }



\begin{document}
\textbf{Some solutions to Kreyszig's
  \textit{Introductory Functional Analysis with Applications}
}

\

\textbf{2.1.6}

Show that in an $n$-dimensional vector space $X$,
the representation of any $x$ as a linear combination of given basis
vectors $e_1, \dots, e_2$ is unique.

\

Suppose we have two representations for $x$:
\begin{align*}
  & x = \alpha_1e_1 + \dots \alpha_n e_n \\
  \text{and } & x = \beta_1e_1 + \dots + \beta_n e_n \\
\iff & 0 = x-x = (\alpha_1-\beta_1)e_1 + \dots + (\alpha_n-\beta_n) e_n 
\end{align*}
Since the basis vectors must be linearly independent,
this means that each $(\alpha_i-\beta_i) = 0$,
i.e. $\alpha_i = \beta_i$.

\

\textbf{2.1.10}

If $Y$ and $Z$ are subspaces of a vector space $X$,
show that $Y \cap Z$ is a subspace of $X$,
but $Y \cup Z$ need not be one. 
Give examples. 

\

For $Y \cap Z$,
we show that this set is closed under scalar multiplication and addition. 
Given $x \in Y\cap Z$,
we know $\alpha x \in Y$ since $Y$ is a vector space,
and $\alpha x \in Z$ since $Z$ is a vector space,
hence $\alpha x \in Y \cap Z$. 
Similarly, 
given any $x$ and $y$ in $Y \cap Z$,
$x+y \in Y$ by definition of subspace for $Y$ and
$x+y \in Z$ by definition of subspace for $Z$,
hence $x+y \in Y \cap Z$. 

For $Y \cup Z$,
we can use $\R^2$ as a counterexample.
Let the ``$x$-axis'' or span\{(1,0)\} act as $Z$ and the
``$y$-axis'' or span\{(0,1)\} act as $Y$.
The union $Y \cup Z$ is not closed under vector addition
(note that $(0,1) \in Y$ and $(1,0) \in Z$): 
$$(1,0) + (0,1) = (1,1) \notin Y \cup Z$$

Clearly the intersection of these two sets, the zero vector, is a subspace. 

\

\textbf{2.2.11}

A subset $A$ of a vector space $X$ is said to be \textit{convex}
if $x,y \in A$ implies
$$ M = \{z \in X : z = \alpha x + (1-\alpha) y, 0 \leq \alpha \leq 1 \} \subset A$$
$M$ is called a closed segment with boundary points $x$ and $y$;
any other $z\in M$ is called an interior point of $M$.
Show that the closed unit ball
$$ \tilde{B}(0;1) = \{x\in X : \lVert x \rVert \leq 1 \}$$
in a normed space $X$ is convex. 

\

We take any two points $x$ and $y$ in the unit ball $\tilde{B}(0;1)$ 
and show that any point in the segment joining $x$ and $y$ is also
in $\tilde{B}(0;1)$.
Let $z \in \tilde{B}(0;1)$.
Then there exists $\alpha \in [0,1]$ such that
\begin{align*}
  z &= \alpha x + (1-\alpha) y \\
  \text{so \quad} \norm{z} &= \norm{ \alpha x + (1-\alpha) y } \\
  \text{triangle inequality: \quad} & \leq \norm{ \alpha x } + \norm{ (1-\alpha) y } \\
  \text{scalar invariance of norm: \quad} &= |\alpha| \norm{ x } + |(1-\alpha)|  \norm{ y } \\
  \text{$\norm{x}, \norm{y} \leq 1$: \quad} &\leq |\alpha| + |1-\alpha| \\
  \text{$0 \leq \alpha \leq 1$: \quad} & = \alpha + 1 - \alpha = 1
\end{align*}
So $\norm{z} \leq 1$ so it must be in the unit ball $\tilde{B}(0;1)$.

\

This is easy enough to see for $\R^2$ but what about for more exotic
spaces like C[0,1]?
This means that if we combine any two continuous functions with a common bound 
in the manner above for $z$,
the result will have the same bound.
See jupyter notebook animation. 

\end{document}
